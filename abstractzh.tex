\begin{CJK*}{UTF8}{bsmi}
{\CJKfamily{bkai}
由於固態硬盤容易發生比特錯誤以及壽命較短的特性,我們通
常采用帶校驗信息塊的獨立硬盤冗余陣列(parity-based RAID)
來提高固態硬盤的可靠性。 我們把小量而隨機的寫入請求稱之
為部分寫入。但部分寫入會增加固態硬盤垃圾回收量以及陣列
校驗塊的更新請求,從而降低陣列的表現及硬盤壽命。因此,
我們提出一個中間件的設計(TWEEN)來提高陣列的寫入效率
並且延長固態硬盤的壽命。TWEEN主要具備以下兩個特性,首
先,它利用日誌式的文件系統(LFS)以及 非易失性隨機存儲器
(NVRAM)來消除部分寫入對於陣列的影響;其次,它對接受
的寫入請求進行分類以達到減少文件系統的垃圾回收的目的。 由
於TWEEN的設計和實現都是基於用戶空間,所以它可以移植到
市面上任何的固態陣列系統。我們利用人造的和現實世界截取的
文件訪問記錄來測試TWEEN的系統表現。相比於 現有的文件系
統,初步測試結果顯示TWEEN可以達到較高的寫入速度,較低
的垃圾回收率以及較長的固態硬盤壽命。}
\end{CJK*}
